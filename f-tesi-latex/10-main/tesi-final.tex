\chapter{Conclusioni}

In questa tesi \e proposto un algoritmo si segmentazione per l'individuazione dei contorni
di macchie cutanee il cui nucleo \e costituito dal metodo di segmentazione basato su un
interessante modello di contorno attivo o {\it snake} proposto da Yezzi et al. in \cite{Yezzi}.
Esso infatti combina i vantaggi dei due approcci con cui vengono affrontati i problemi di
segmentazione: quello basato sulla propriet\a di omogeneit\a dei segmenti e quello che
realizza la segmentazione individuando i bordi che separano i segmenti.

Per ottenere inoltre una rappresentazione compatta dello stesso contorno, in vista delle
successive misure, si \e scelta la sua rappresentazione a B-splines cubiche uniformi.
Visto per\o che la forma dell'oggetto da segmentare non \e nota a priori vi \e la necessit\a
di definire un metodo per permettere allo snake di adattarsi alla diversa complessit\a
della macchia da individuare.
Si \e perci\o considerato un metodo che sfrutta la riparametrizzazione della curva a partire
dai campioni ottenuti dall'intersezione con un reticolo, che \e la scomposizione simpliciale 
(o triangolazione) del piano immagine, e la successiva ripartizione di tali punti fra
i nuovi {\it control points}, in funzione di un criterio di complessit\a che tiene conto della
distanza e del grado di allineamento, o "curvatura", fra i campioni.

Dai test effettuati si \e visto che l'algoritmo \e poco sensibile
all'inizia-lizzazione del contorno attivo, comunque per ridurre i tempi di calcolo \e opportuno
che la curva iniziale sia prossima al bordo dell'oggetto da segmentare.
Per questo si \e
definita una procedura di inizializzazione basata sul contorno della regione compatta ottenuta
dalla binarizzazione, e successiva elaborazione morfologica, della componente principale della
{\it trasformata di Karhunen-Lo\`eve} dell'immagine originale.

In relazione ai tempi di esecuzione si pu\o dire che il punto critico \e dato dal calcolo delle
medie interne ed esterne alla curva, \e quindi opportuno considerare con attenzione una 
possibile approssimazione della funzione intensit\a delle primitive.

\`E stato definito inoltre un operatore locale basato su una particolare misura di anisotropia
per ridurre l'effetto di disturbo dei peli, non sempre 
trascurabile nel momento in cui tali elementi si trovano in prossimit\a del bordo e quindi
possono influenzare la dinamica dello {\it snake}.

L'algoritmo \e stato sviluppato per la segmentazione di macchie cutanee, ma come \e
strutturato e per la natura del modello di snake \e potenzialmente applicabile ad altri
oggetti non appena siano caratterizzabili con un opportuno vettore di primitive ({\it features}).

\finepar