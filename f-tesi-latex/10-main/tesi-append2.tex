\appendice[Funzionali definiti su regioni regolari]
          {Gradiente di funzioni integrali definite su regioni regolari}\lb{B}

%========================================================================================
%\section{}

\blem
Data ${\bf A}$, matrice quadrata $(n \times n)$, ${\bf x}$ e ${\bf y}$ due vettori colonna
$n-dim$ allora vale
 \be
 {\bf A}\,{\bf x}\cdot{\bf y}\,-\,{\bf A}\,{\bf y}\cdot{\bf x}\,=\,
             {\bf x}\cdot\Big[{\bf A}^T-{\bf A}\Big]\,{\bf y}.
 \ee
\lb{lemmaB1}
\elem

\begin{proof}
  Dalla definizione di prodotto scalare $(\cdot)$  
 $$
 {\bf A}\,{\bf x}\cdot{\bf y}\,=\,({\bf A}\,{\bf x})^T\,{\bf y}
 $$
 e dalla commutativit\a si ha che 
 $$
 {\bf A}\,{\bf x}\cdot{\bf y}\,-\,{\bf A}\,{\bf y}\cdot{\bf x}\,=\,
 {\bf x}^T\,{\bf A}^T\,{\bf y}\,-\,{\bf x}^T\,{\bf A}\,{\bf y}\,=\,
 $$
 $$ 
 {\bf x}^T\,\Big[{\bf A}^T-{\bf A}\Big]\,{\bf y}\,=\,
 {\bf x}\cdot\Big[{\bf A}^T-{\bf A}\Big]\,{\bf y}. 
 $$
\end{proof}


Sia
\be 
h(t)\,=\,\ds\int_{R(t)}\,f(x,y)\,dxdy
\ee
con $R(t)$ la regione racchiusa dalla curva chiusa $\Gamma(t)$ e $f:\M(R)^2 \longrightarrow \M(R)$
una funzione continua.

Definita la funzione vettoriale {\bf g}(x,y) con
\be
g_1(x,y)\,=\,-\,\ds\int_0^y\,f(x,\lambda)\,d\lambda \qquad
g_2(x,y)\,=\,\ds\int_0^x\,f(\lambda,y)\,d\lambda
\ee
\e possibile scrivere
\be
h(t)\,=\,\ds\int_{R(t)}\,\frac{1}{2}\,
 \Bigg[\frac{\partial g_2}{\partial x}\,(x,y)\,-\,\frac{\partial g_1}{\partial y}\,(x,y)\Bigg]
 \,dxdy
\ee

Dal {\it teorema di Stokes}, che nel caso in esame di curve piane si riduce alle {\it formule
di Green}, si ottiene 
\be 
h(t)\,=\,\frac{1}{2}\ds\int_{\Gamma(t)}\,{\bf g}\cdot{\bf \tau}\,dl.
\ee
Considerando che $\Gamma(t)$ \e rappresentata dalla carta ${\bf r}(s,t)=[x(s,t),y(s,t)]$,
con $s \in [0,1]$\footnotemark, e che il vettore tangente alla curva \e esprimibile in funzione
della parametrizzazione
\be
{\bf \tau}(s,t)\,=\,\frac{{\bf r}_s(s,t)}{\|{\bf r}(s,t)\|}
\lb{tau}
\ee
risulta
\be
h(t)\,=\,\frac{1}{2}\ds\int_0^1\,{\bf g}\cdot\frac{{\bf r}_s(s,t)}{\|{\bf r}(s,t)\|}\,
         {\|{\bf r}(s,t)\|}\,ds\,=\,
       \,\frac{1}{2}\ds\int_0^1\,{\bf g}\cdot{\bf r}_s(s,t)
\ee

\footnotetext{Si \e scelto il dominio del parametro $s$ in $[0,1]$ per comodit\aac; \e
comunque valido un qualsiasi intervallo del tipo $[\omega_o,\omega_1]$.} 

Calcolare $\nabla h$ lungo la curva regolare $\Gamma$ \e equivalente a massimizzare la
variazione $h^{\,\prime}(t)$
\be
h^{\,\prime}(t)\,=\,\frac{1}{2}\ds\int_0^1\,\Bigg[{\bf g}_t\cdot{\bf r}_s(s,t)\,+\,
                                            {\bf g}\cdot{\bf r}_{s,t}(s,t)\Bigg]\,ds.
\ee

Integrando per parti rispetto a $s$ il secondo termine dell'integranda risulta
\beqa
\ds\int_0^1\,{\bf g}\cdot{\bf r}_{s,t}(s,t)
  & = & {\bf g}\cdot{\bf r}_t\Big\vert_{s=0}^{s=1}\,
        -\,\ds\int_0^1\,{\bf g}_s\cdot{\bf r}_t(s,t)\,ds\,= \nonumber\\
  & = & -\,\ds\int_0^1\,{\bf g}_s\cdot{\bf r}_t(s,t)\,ds 
\eeqa
in quanto il primo addendo \e nullo dato che la curva \e chiusa e quindi 
${\bf r}(0)={\bf r}(1)$ per cui ${\bf g}({\bf r}(0))={\bf g}({\bf r}(1))$ e cos\iac\,pure 
${\bf r}_t(0)={\bf r}_t(1)$.

Considerando inoltre che il differenziale di funzioni composte
\beqa
{\bf g}_s & = & {\bf g}_s({\bf r}(s,t))\,=\,J{\bf g}(x,y){\bf r}_s(s,t) \nonumber \\
          &   & \\
{\bf g}_t & = & {\bf g}_t({\bf r}(s,t))\,=\,J{\bf g}(x,y){\bf r}_t(s,t), \nonumber
\eeqa
con
$$
J{\bf g}(x,y)\,=\,
\smatrix{2}{{\ds\frac{\partial g_1}{\partial x}} & {\ds\frac{\partial g_1}{\partial y}} \cr
        {\ds\frac{\partial g_2}{\partial x}} & {\ds\frac{\partial g_2}{\partial y}}}(x,y)\,=\,
\smatrix{2}{ 0 & -f \cr
             f &  0 }(x,y)\,=\,
$$
\be
=\,f(x,y)\,\smatrix{2}{ 0 & -1 \cr
                        1 &  0 },
\lb{jac}
\ee
assieme alle precedenti si ottiene l'espressione per $h^{\,\prime}(t)$
\be
h^{\,\prime}(t)\,=\,\frac{1}{2}\ds\int_0^1\,\Big[J{\bf g}\,{\bf r}_t\cdot{\bf r}_s\,-\,
                    J{\bf g}\,{\bf r}_s\cdot{\bf r}_t\Big]\,ds.
\ee

Applicando il Lemma \r{lemmaB1} con ${\bf x}={\bf r}_t$, ${\bf A}=J{\bf g}$ e
${\bf y}={\bf r}_s$ 
\be
h^{\,\prime}(t)\,=\,\frac{1}{2}\ds\int_0^1\,
                  \Big[{\bf r}_t\cdot\bigg((J{\bf g})^T-J{\bf g}\bigg)\,{\bf r}_s\Big]\,ds
\lb{hprime}
\ee

Dalla (\r{jac}) si vede che lo {\it Jacobiano} \e antisimmetrico, per cui lo \e pure la
differenza $(J{\bf g})^T-J{\bf g}$ che infatti vale
$$
2\,f(x,y)\,\smatrix{2}{  0 & 1 \cr
                        -1 & 0 },
$$
dato che per la propriet\a di antisimmetria $(J{\bf g})^T=-J{\bf g}$.

Sostituendo nella (\r{hprime}) 
\be
h^{\,\prime}(t)\,=
         \,\ds\int_0^1\,\Big[{\bf r}_t\cdot\,f\,\smatrix{2}{  0 & 1 \cr
                                                             -1 & 0 }\,{\bf r}_s\Big]\,ds
\lb{hprime2}
\ee

Nell'ipotesi per cui l'orientamento positivo di percorrenza di $\Gamma$ sia quello antiorario
e che l'orientamento positivo della regione $R$ sia ${\bf e}_3$ allora sussiste la relazione
fra il versore tangente e quello normale alla curva 
$$
{\bf n}\,=\,\smatrix{2}{  0 & 1 \cr
                         -1 & 0 }\,{\bf \tau};
$$
considerando inoltre l'espressione (\r{tau}) si ha
\be
\|{\bf r}_s\|\,{\bf n}\,=\,\smatrix{2}{  0 & 1 \cr
                                        -1 & 0 }\,{\bf r}_s.
\ee 

Quest'ultimo \e uno dei termini del prodotto scalare dell'integranda in (\r{hprime2}) e 
quindi
\be
h^{\,\prime}(t)\,=
         \,\ds\int_0^1\,\Big[{\bf r}_t\cdot\,f\,{\bf n}\Big]\,\|{\bf r}_s\|\,ds\,=\,
         \,\ds\int_{\Gamma(t)}\,{\bf r}_t\cdot\,f\,{\bf n}\,dl
\lb{hprime3}
\ee
per cui la direzione lungo la quale deformare la curva $\Gamma$ in modo che sia massima
$h^{\,\prime}$ \e data dal flusso 
$$
{\bf r}_t\,=\,f\,{\bf n}.
$$


\finepar