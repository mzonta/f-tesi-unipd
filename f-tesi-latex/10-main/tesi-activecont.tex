\chapter{Contorni attivi}\lb{CA}

Nel Capitolo \r{SGM} si \e accennato all'utilizzo di alcuni operatori locali che permettono di
individuare nell'immagine delle opportune primitive ({\it features}) il cui insieme costituisce una nuova
immagine o mappa delle primitive ({\it features map}) la cui interpretazione pu\o permettere di
individuare gli elementi (oggetti) che compongono l'immagine.

Il problema \e che, tranne in casi molto fortunati in cui si hanno immagini costituite da
elementi distinti, prive di particolari disturbi e con superfici dei materiali ripresi sostanzialmente uniformi,
vi possono essere diverse false {\it features}.
Occorre pertanto definire un criterio che permetta di estrarre solo quelle vere.
Nel caso pi\u semplice si pu\o considerare una funzione costante che coincide con un
valore di soglia, valore che per\o pu\o non essere semplice individuare e che inoltre pu\o generare
il problema opposto, ovvero portare ad eliminare anche alcune {\it features} buone che per\o
sono "deboli", per cui si ottiene una descrizione frammentaria della {\it features map} 
dell'immagine che ne complica la successiva elaborazione.

Come gi\a accennato precedentemente, una soluzione di notevole interesse 
\e data dagli {\it active contours}, chiamati comunemente {\it snakes}\footnotemark \cite{Kass}.

\footnotetext{La famiglia degli {\it active contours} \e costituita oltre che da {\it snakes},
che ne solo la forma pi\u semplice, anche da {\it deformable template} e {\it dynamic contours}\cite{BlakeAC}}.


%======================================================================================
\section{Snakes}

Con {\it snake} si intende una curva deformabile ${\bf r}(s)$, $0 \leq s \leq 1$,
che sotto l'azione di opportune "forze" definite nello spazio delle primitive 
dell'immagine tende a deformarsi in modo da minimizzare una "energia" o
"funzione costo".
Nel caso particolare della soluzione di problemi di segmentazione tale funzione
\e definita in modo che la condizione di minimo porti la curva ad adagiarsi lungo
i contorni degli oggetti, per cui lo spazio delle primitive \e dato generalmente dalla 
mappa degli {\it edgles}.

\boss
Determinare la mappa delle features completa pu\o richiedere tempi eccessivi per
l'applicazione considereta, ed inoltre non \e detto sia necessario; per cui tali
primitive possono essere individuate limitatamente all'intorno della curva analizzando
l'immagine lungo direzioni particolari, tipicamente lungo la normale alla curva
\cite{BlakeAC}.
\lb{ossploc}
\eoss

L'energia totale ${\cal E}({\bf r})$, lungo tutta la curva, che individua lo stato dello
snake \e composta di due termini
\be
{\cal E}({\bf r})\,=\,{\cal E}_s({\bf r})\,+\,{\cal P}({\bf r}).
\lb{ergtot}
\ee
\ben
\im
Il primo rappresenta l'{\it energia interna} che controlla la regolarit\a della curva
e quindi un vincolo sulla classe delle forme che pu\o segmentare:
\be
{\cal E}_s({\bf r})\,=\,\int_0^1\,w_1(s)|{\bf r}_s|^2\,+
                                \,w_2(s)|{\bf r}_{ss}|^2\,ds,\quad (\footnotemark)
\ee 

\footnotetext{Con $f_x$ si denota la derivata di $f$ rispetto alla variabile $x$.}

dove $w_1(s)$ controlla l'"elasticit\a" rispetto alle sollecitazioni che tendono a dilatare
o comprimere lo snake (trazione e compressione) e $w_2(s)$ la "rigidezza" rispetto alle 
deformazioni trasversali allo snake (flessione).

Infatti se si aumentasse solo $w_1(s)$, minimizzare l'energia corrisponderebbe ad accorciare
la lunghezza della curva fino, caso limite, a ridurla ad un punto; nel caso di $w_2(s)$
invece si avrebbe la tendenza ad aumentare la regolarit\a della curva fino ad approssimare
una linea retta. Al contrario riducendo i due termini, al limite ponendoli a zero\footnotemark,
\e possibile ottenere una curva che presenti dei punti angolosi o addidrittura delle
discontinuit\a.

\footnotetext{I coefficienti $w_1(s)$ e $w_2(s)$ sono non negativi.}

\im
Il secondo termine della (\r{ergtot}) rappresenta l'{\it energia potenziale} dello {\it snake}
rispetto allo spazio delle primitive; per cui se $\cal P$ \e definita in modo tale
che i suoi punti di minimo siano in corrispondenza delle {\it features}, allora la minimizzazione
della (\r{ergtot}) spinge lo snake ad approssimarle.
L'energia potenziale lungo tutta la curva pu\o essere definita come l'integrale curvilineo
di una funzione potenziale 
\be
{\cal P}({\bf r})\,=\,\int_0^1\,P({\bf r}(s))\,ds
\ee
dove $P({\bf r})$ \e una funzione che mappa i punti dell'immagine nello spazio delle primitive.
Alcuni esempi tipici delle definizioni di $P({\bf r})$ sono
 \ben
 \im $P({\bf r})\,=\,\pm\,c\,\bigg(\,G_\sigma \ast I\,({\bf r})\,\bigg)$
     se si considera come primitiva i picchi di intensit\aac: bassa o alta intensit\aac, chiaro o
     scuro, a seconda del segno;
 \im $P({\bf r})\,=\,-\,c\,\Big|\,\nabla\bigg(\,G_\sigma \ast I\,({\bf r})\,\bigg)\,\Big|$
     in questo caso si considerano gli {\it edgels}.
 \een
Per entrambi i casi si sono considerati solo i punti della curva, ${\bf r}(s)$, in base
all'osservazione \r{ossploc}.

Come accennato nel caso degli {\it edge-detector}, l'azione della convoluzione con la gaussiana 
$G_\sigma(x,y)$ \e rivolto sia alla riduzione dell'$SNR$ all'uscita del filtro $P(x,y)$
sia a migliorare il grado di localizzazione degli {\it edgels}.
Nel caso degli {\it snakes} questo pu\o essere visto come una trasformazione della superficie
su cui esso si muove, l'immagine, che possa da un lato facilitare l'avvicinamento ai punti
del contorno ($\sigma$ grande) e dall'altro fare in modo che l'errore di approssimazione
ammesso (in termini di localizzazione del bordo) sia piccolo ($\sigma$ piccola).
Anche in questo caso le due condizioni lavorano in direzioni opposte, per cui
risulta non sempre facile fissare un valore per tutte le zone dell'immagine;
eventualmente si potrebbe considerare $\sigma$ variabile in base alle caratteristiche
locali dell'immagine stessa come nel caso degli operatori anisotropi.
\een

%-------------------------------------------------------------------------------------
\section{Un approccio Lagrangiano}

La soluzione al problema della minimizzazione dell'energia $\cal E$ pu\o essere determinata
in modo statico calcolando direttamente i punti di estremo; in realt\aac, anche in considerazione
del modo in cui vengono determinate le primitive, \e conveniente considerare lo {\it snake}
come una curva che si muove sul piano immagine ({\it dynamic snake}) la cui posizione e forma
variano ad ogni passo in relazione alle misure effettuate; ne deriva che la 
condizione di minimo dell'energia coincide perci\o con lo stato di equilibrio del sistema
dinamico.

Il nuovo modello \e ottenuto ridefinendo la carta ${\bf r}(s,t)$ per tener conto della 
variazione nel tempo $t$ e considerando che la curva abbia una densit\a di massa $\mu (s)$,
per cui \e presente anche una componente di energia cinetica
\be
{\cal T}({\bf r})\,=\,\frac{1}{2}\int_0^1\,\mu (s)|{\bf r}_t|^2\,ds
\ee
A causa dell'inerzia introdotta da $\mu (s)$, affinch\'e il sistema raggiunga uno stato
stabile \e necessario introdurre un termine a cui sia affidata la dissipazione dell'energia
cinetica immagazzinata.
A tal scopo si pu\o considerare la funzione {\it dissipazione di Rayleigh} definita
\be
{\cal D}({\bf r}_t)\,=\,\frac{1}{2}\int_0^1\,\gamma |{\bf r}_t|^2\,ds.
\ee
Applicando il {\it principio di Hamilton} si perviene alle {\it equazioni di Lagrange}
\cite{BlakeAV1}
\be
\mu {\bf r}_{tt}\,+\,\gamma {\bf r}_t\,-\,\frac{\partial}{\partial s}\,(w_1{\bf r}_s)\,+
 \,\frac{\partial^2}{\partial s^2}\,(w_2{\bf r}_{ss})\,=\,-\nabla P({\bf r}(s,t))
\lb{eqlagr}
\ee
in cui si sono considerate $\mu (s)=\mu$ e $\gamma (s)=\gamma$ costanti lungo tutta
la curva.
     
%....................................................................................
\subsubsection{Kalman-snakes}

Tenendo conto della dinamica dello snake \e possibile risolvere problemi pi\u complessi
in cui la mappa delle {\it features} varia nel tempo a cusa del moto relativo tra il sensore
e l'oggetto da inseguire: si parla perci\o di {\it data tracking}.
In questo caso l'equilibrio raggiunto \e dinamico in quanto lo snake tende ad approssimare
dei dati che non sono costanti nel tempo.
Le prestazioni del sistema migliorano sensibilmente se si integrano nel tempo le misure
delle primitive con il modello dinamico dello snake utilizzando il {\it filtro di Kalman},
il cui passo di predizione permette di stabilire come potr\a essere l'evoluzione dello
snake al passo successivo individuando la regione di confidenza entro cui sar\a probabile
trovarlo nell'immagine succesiva.
La posizione reale verr\a individuata dalle misure effettuate nel passo di aggiornamento.

L'introduzione del filtro, il cui peso computazionale \e generalmente trascurabile,
permette di ridurre la regione dell'immagine su cui effettuare le misure, che \e la
componente critica per la quantit\a di dati considerati, e quindi migliorare l'efficienza
generale dell'algoritmo.

%---------------------------------------------------------------------------------------
\section{Contorni dinamici}

Nel modello utilizzato finora gli unici vincoli sulle deformazioni a cui \e soggetto lo snake
sono dati dai termini di energia interna associati ai coefficienti $w_1(s)$ e $w_2(s)$.

Esiste la possibilit\a di vincolare la curva in modo che assuma delle forme che si discostino
relativamente da un modello prefissato ({\it template}).
La curva \e quindi definita da una carta ${\bf r}(s;{\bf X})$ \cite{BlakeAC}
che \e funzione di un vettore di parametri che tengono
conto delle possibili trasformazioni che questa pu\o subire (rotazioni, traslazioni,
dilatazioni, etc.); ovvero a partire da una forma di
riferimento si applicano le trasformazioni, scelte fra quelle accettabili, che permettono 
di sovrapporla all'oggetto corrispondente che si trova sul piano immagine:
si parla perci\o di {\it deformable templates}.

Considerando quindi il modello dinamico definito dai {\it Kalman snakes} proiettato
nello spazio delle forme ({\it deformable templates}) si ottengono i {\bf contorni dinamici}
o {\bf dynamic contours}.

%========================================================================================
\section{Discretizzazione dello snake}

Le soluzioni dell'equazione (\r{eqlagr}) sono curve continue nello spazio $\M(R)^2$, mentre
per poterle elaborare numericamente \e necessario discretizzarle.
Il modo pi\u semplice \e dato dal campionamento della curva con un passo di campionamento
$h=1/(N-1)$ nello spazio del parametro $s\in[0,1]$ per cui i campioni della curva sono 
${\bf r}(i\,h)$ con $i=1,\dots,N$.
L'equazione differenziale si trasforma perci\o in un'equazione alle differenze sostituendo
le derivate prima e seconda con le espressioni \footnotemark
\be
{\bf r}_s(s_i)\,=\,\frac{{\bf r}(s_i)\,-\,{\bf r}(s_{i-1})}{h} \quad e \quad
{\bf r}_{ss}(s_i)\,=\,\frac{{\bf r}(s_{i+1})\,-2\,{\bf r}(s_i)\,+\,{\bf r}(s_{i-1})}{h^2}
\ee
Analogamente per le derivate rispetto la variabile temporale $t$.

\footnotetext{Discretizzazione con metodo di Eulero implicito.}

L'equazione di Lagrange (\r{eqlagr}) viene espressa da una equazione differenziale
del secondo ordine \cite{BlakeAV1}
\be
{\bf M\,\ddot{u}}\,+\,{\bf C\,\dot{u}}\,+\,{\bf K\,u}\,=\,{\bf f} \quad con \quad
{\bf u}=\{{\bf u}_i\}_{i=1,\dots,N}={\bf r}(i\,h)
\ee
dove ${\bf M}$ \e la matrice delle masse, ${\bf C}$ tiene conto dei fenomeni di dissipazione,
${\bf K}$ della rigidezza dello snake e ${\bf f}$ dipende da $\nabla P$.

Effetto del campionamento della curva \e la perdita della propriet\a di continuit\a 
e quindi dell'informazione sulla forma individuata dallo {\it snake} stesso, per cui,
sia nel caso particolarmente interessante dei contorni dinamici accennati precedentemente
sia nel caso dei semplici {\it snake} nella forma data dall'equazione (\r{ergtot}), si ricorre
ad approssimazioni con metodi agli "{\it elementi finiti}" in cui la curva ${\bf r}$ \e
definita dalla combinazione lineare di particolari funzioni $B_i$, elementi della base di un
opportuno spazio vettoriale, pesate dai {\it nodi} ${\bf q_i}(t)$
\be
{\bf r}(s,t)\,=\,\sum_{i=1}^N\,B_i(s)\,{\bf q}_i(t)
\ee
Un modello particolarmente interessante, che sar\a considerato nel nostro caso, \e dato
dall'approssimazione a {\bf splines} in cui la curva \e approssimata da funzioni polinomiali
a tratti parametrizzate dai nodi ${\bf q_i}(t)$ che prendono il nome di {\it control points} \cite{BlakeAV2} \footnotemark. 

\footnotetext{Altre descrizioni parametriche del contorno possono essere ricavate utilizzando i coefficienti
di Fourier.}

%========================================================================================
\section{Un approccio statistico}

L'introduzione degli {\it snakes} ha permesso di risolvere alcuni problemi che affliggono i metodi
di segmentazione che si basano sulla sola {\it edge-detection} fornendo un modello che permette
di mantenere le informazioni sulla forma degli oggetti a cui appartengono gli {\it edgels}.

Si \e quindi posto il problema come la minimizzazione di una particolare funzione
energia, dove il termine legato all'immagine \e definito da un potenziale, che \e minimo
in corrispondenza dei bordi dei segmenti.

Finora si \e considerato il potenziale definito in funzione delle propriet\a locali 
nell'intorno della curva ({\it feautures}) e in particolare considerando il gradiente dell'immagine
opportunamente filtrata; ma, a causa dell'operatore differenziale, 
tali misure sono molto sensibili ai disturbi di diversa natura
che possono essere presenti nell'immagine stessa, ovvero
tale sensibilit\a pu\o essere ridotta con un aumento
dell'azione filtrante ma a scapito della capacit\a di localizzazione.

Per ovviare a tale limite sono state proposte soluzioni alternative che 
tengono conto anche delle propriet\a di omogeneita dei segmenti.
Si possono individuare due strade:
\ben
\im si identificano i bordi in modo indiretto, senza utilizzare operatori differenziali,
    ridefinendo un'opportuna energia che tenga conto anche della statistica dei segmenti
    e forzando i contorni attivi a disporsi in modo da massimizzare la differenza fra
    i segmenti separati dai contorni stessi \cite{Yezzi}, \cite{Yuille};
\im considerando i due approcci distinti che interagiscono fra di loro in base ad opportune
    regole \cite{Duncan} che stabiliscono come il risultato di uno venga valutato
    dall'altro e quindi la possibile contromossa pi\u conveniente in termini di "costo",
    fino a giungere ad una condizione di equilibrio.
\een

%---------------------------------------------------------------------------------------
\subsection{L'algoritmo di Yezzi, Tsai e Willsky}\lb{AYTW}

Nell'ambito delle soluzioni del primo tipo vi \e l'algoritmo proposto da A. Yezzi, A. Tsai
e A. Willsky in \cite{Yezzi} che permette di incorporare in un nuovo modello di {\it snake}
sia informazioni locali sia globali attraverso una funzione energia diversa da quelle
considerate finora.
Il vincolo principale nell'utilizza-re il modello proposto \e che siano note a priori 
come le primitive scelte per la segmentazione descrivono i diversi elementi dell'insieme
da segmentare, per cui ad ogni oggetto sia associabile un valore per ogni feature.
\boss
Il valore non deve essere necessariamente noto ma ,se esiste, si deve poter dire che
allora esso contraddistingue quell'oggetto rispetto agli altri.
\eoss
Per esempio si possono avere diversi oggetti che sono contraddistinti da colori diversi,
da tessitura diversa, etc.
Anche se i colori non sono noti si sa che comunque sono differenti.

In particolare si considera l'algoritmo nel caso di immagini bimodali o
binarie, cio\e costituite da due regioni ({\it foreground} e {\it background}) rispetto alla
primitiva considerata: il {\it colore}.

\balg {\bf Flusso per immagini binarie.}\par
Sia $I(x,y)$ l'immagine e $F \in I$ ({\it foreground}) la regione obiettivo della 
segmentazione, per cui $B=F^c$ ({\it background}) \e il suo complementare; lo scopo
\e quindi deformare la {\it curva chiusa} iniziale $\Gamma (t)$ definita su $I$ in modo che
tenda ad abbracciare la regione $F$ e quindi coincida con
il bordo $\partial F$.

Definiamo $P$ la funzione $N$ dimensionale che definisce il vettore di primitive scelta fra
quelle che meglio discriminano le due regioni; si considerino quindi le medie $m^{int}$ e $m^{est}$
dei valori di $P$ nelle due regioni rispettivamente interna $R^{int}$ ed esterna $R^{est}$
rispetto alla curva:
\be
m_j^{int}(\Gamma)\,=\,\frac{S_j^{int}(\Gamma)}{A^{int}(\Gamma)} \quad 
                     j\,=\,1,\dots,N \quad (\footnotemark)
\lb{med_int}
\ee   
\footnotetext{Il pedice $j$ indica la componenete del vettore delle primitive considerate;
si noti che $A$ \e comunque fissa in quanto dipende solo da $\Gamma$.}
$$
S_j^{int}\,=\,{\ds\int_{R^{int}}}\,P_j({\bf x})\,d{\bf x} \qquad\qquad
A^{int}\,=\,{\ds\int_{R^{int}}}\,d{\bf x}
$$

Analogamente per $m_e(\Gamma)$.

La nuova funzione energia \e definita in funzione della distanza fra $m^{int}$ e $m^{est}$
\be
{\cal E}\,=\,-\,\frac{1}{2}\,\|m^{int}-m^{est}\|^2 
\lb{EYezzi}
\ee

{\sl in cui si \e considerata la norma in quanto in generale $P$ \e una funzione vettoriale;
questo \e molto importante perch\e permette di combinare in modo semplice caratteristiche
dell'immagine diverse tra loro}.

Tale energia \e minima quando la differenza fra le due medie \e massima, ovvero quando
la curva coincide col bordo\footnotemark.
\footnotetext{In pratica non si arriva ad una perfetta separazione, se non in casi particolari,
per cui si intende che il gradiente di variazione dell'energia \e inferiore ad un certo limite
o, equivalemtemente, che si possa ritenere costante.}

La curva iniziale potr\a essere prossima al bordo di $F$ ma non coincidere e quindi o
intersecare le due regioni o essere contenuta in una delle due regioni, per cui la distanza
fra i due vettori di media sar\a inferiore al valore finale.

L'evoluzione della curva pu\o essere ricavata considerando 
\beqa
\frac{d\Gamma}{dt} & = & -\,\nabla {\cal E} \quad (\,\footnotemark\,)\\
                   & = & \sum_{j=1}^N\,(m_j^{int}-m_j^{est})\,
                                       \bigg[\nabla m_j^{int}-\nabla m_j^{est}\bigg] \nonumber
\eeqa
\footnotetext{Con $\nabla {\cal F}$ si intende la direzione del gradiente di ${\cal F}$ 
 definito sullo spazio della curva $\Gamma$; ovvero la direzione lungo la quale una 
 deformazione della curva $\Gamma$ massimizza la variazione di ${\cal F}$}

dove dalla definizione data per $m^{int}$ e $m^{est}$ nella (\r{med_int}) si ricava
\beqa
\nabla m_j^{int} & = & \frac{A^{int} \nabla S_j^{int}-S_j^{int} \nabla A^{int}}{{A^{int}}^2}\,=\,
                 \frac{\nabla S_j^{int}-m_j^{int} \nabla A^{int}}{A^{int}} \\
\nabla m_j^{est} & = & \frac{\nabla S_j^{est}-m_j^{est} \nabla A^{est}}{A^{est}} \nonumber
\eeqa
con
\be
\nabla S_j^{int}\,=\,P_j\,{\bf n} \qquad \qquad 
\nabla A^{int}\,=\,{\bf n}
\ee
in cui {\bf n} \e la normale esterna alla curva $\Gamma$ (si veda l'Appendice \r{B}).

Allora risulta
\beqa
\nabla m_j^{int} & = & \frac{P_j-m_j^{int}}{A^{int}}\,{\bf n} \\
\nabla m_j^{est} & = & -\,\frac{P_j-m_j^{est}}{A^{est}}\,{\bf n} \nonumber
\eeqa
Il segno meno nell'ultima equazione deriva dal fatto che la normale esterna alla curva
vista rispetto $R^{int}$ \e la normale interna rispetto $R^{est}$.

Quindi riassumendo
\be
\frac{d \Gamma}{dt}\,=\,\sum_{j=1}^{N}\,(m_j^{int}-m_j^{est})\,
                 \bigg[\frac{P_j-m_j^{int}}{A^{int}}\,+\,\frac{P_j-m_j^{est}}{A^{est}}\bigg]
\lb{flow}
\ee
\ealg
 
\boss
Data la definizione (\r{EYezzi}), nel caso particolare di una sola regione $F$ connessa, 
l'energia ${\cal E}$ pu\o assumere un solo valore di minimo e un certo numero di massimi
che coincidono con ${\cal E}=0$ e che corrisponde a $m^{int}=m^{est}=(P(F)+P(B))/2$ e che
si verifica se la {\it curva iniziale} \e tale per cui suddivide le due regioni $F$ e $B$
in due parti con aree uguali. 
Dalla (\r{flow}) si conclude che questa condizione non permette l'evoluzione dello snake che
rimane nella configurazione iniziale; per evitare una tale situazione si pu\o verificare 
all'inizio dell'evoluzione se $m^{int}=m^{est}$ ed eventualmente introdurre una qualsiasi 
perturbazione su $\Gamma$.
\eoss

\finepar



