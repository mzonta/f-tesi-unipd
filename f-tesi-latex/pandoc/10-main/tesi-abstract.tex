% Sommario
\preface

%--------------------------------------------------------------------------------------------
In questa tesi ci si pone l'obiettivo di fornire una soluzione al problema della segmentazione
di macchie cutanee, i nei, per effettuare successivamente delle misure che possono riguardare
lo stato dei bordi, la forma e le simmetrie.

Il problema della segmentazione \e risolto utilizzando un algoritmo basato su un nuovo modello
di contorni attivi, {\it snakes} o {\it active contours}, in cui la funzione energia
potenziale, o funzione costo, non \e legata alla definizione del bordo come gradiente di una
immagine intensit\aac, ma \e funzione della distanza fra le intensit\a interna ed esterna
alla curva chiusa che individua lo snake.
Si realizza cos\iac\,\,la fusione delle informazioni locali nell'intorno dello snake e quelle
globali date dalla media delle intensit\aac\,\,nelle due regioni individuate.

Per quanto riguarda il modello del contorno si \e considerata la sua rappresentazione a
B-splines cubiche uniformi, la quale permette di mantenere la curva sufficientemente regolare
con una sua descrizione sufficientemente compatta; infatti essa \e descritta interamente da
un insieme limitato di parametri: i suoi {\it control points}.
Per permettere alla curva di adattarsi alla complessit\a dell'oggetto da segmentare, che per
sua natura non \e nota a priori, \e necessario rendere flessibile il numero dei control points;
\e stato perci\o definito un metodo di ridistribuzione dei campioni della curva con l'ausilio
di un reticolo, che rappresenta una scomposizione simpliciale (triangolazione) del piano
immagine.
L'associazione tra i nuovi campioni e i nuovi control points \e realizzata in base ad un
opportuno criterio che misura la complessit\a locale della curva nei punti campione
individuati.

Per migliorare le prestazioni della segmentazione \e opportuno partire da un contorno
iniziale prossimo a quello reale della macchia considerata, intendendo quindi tale
procedimento come una fase di presegmentazione.
Lo stato iniziale dello snake pu\o essere dato dal contorno della regione ottenuta a partire
dalla binarizzazione della componente principale della {\it trasformata di Karhunen-Lo\`eve}
(o di {Hotelling}) dell'immagine originale, successivamente elaborata con degli operatori
morfologici in modo che risulti compatta.

Elaborando la componente principale con un operatore locale anisotropo, il cui
comportamento in un punto dell'immagine dipende dalle caratteristiche dei punti in un
opportuno intorno, \e possibile definire una procedura per ridurre l'effetto di disturbo
di elementi come i peli.