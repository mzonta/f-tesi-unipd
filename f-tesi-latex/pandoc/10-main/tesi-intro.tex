% Introduzione
\chapter{Introduzione}
%---------------------------------------------------------------------------------------------
Uno dei problemi posti in dermatoscopia \e la misura delle caratteristiche delle macchie
cutanee, per cui \e di interesse sviluppare degli strumenti che permettano di effettuare
tali misure in modo automatico.

Le misure riguardano generalmente le dimensioni, lo stato del bordo, se \e pi\u o meno
frastagliato, e le simmetrie sia relativamente alla forma sia rispetto alla pigmentazione
della superficie del neo.

Per poter effettuare tali misure \e necessario, per prima cosa, individuare la macchia, o neo,
estraendola dal contesto determinato dalla pelle circostante, tenendo presente gli elementi di
disturbo tipo i peli, le cui dimensioni non possono essere considerate sempre trascurabili
relativamente a quelle del neo.

\vs(5)

Due sono quindi gli obiettivi che ci si \e proposti di raggiungere:
\ben
\im un problema di {\it segmentazione}: individuare il neo;
\im rappresentazione semplice e compatta del suo contorno per facilitare le successive misure.
\een

Per quanto riguarda la segmentazione esiste un'ampia scelta di metodi che si sviluppano a
partire da due modi equivalenti di interpretare il problema.
I primi definiscono la soluzione come la partizione dell'immagine in regioni uniformi, ovvero
la partizione a cardinalit\a massima ottenibile rispetto ad un dato criterio di uniformit\aac,
che nel nostro caso potrebbe essere il colore (in quanto il neo \e una regione con
pigmentazione differente, generalmente pi\u scura, rispetto alla pelle circostante).
La seconda classe di metodi invece considera come obiettivo l'individuazione dei contorni che
separano le diverse regioni.

Concettualmente le due tecniche sono equivalenti, in pratica, hanno un comportamento
complementare: nel primo caso si ha una minor sensibilit\a al rumore presente
nell'intorno dei contorni mentre si ha difficilmente una rappresentazione compatta dei
segmenti ottenuti; nel secondo caso invece si ha una maggior sensibilit\a al rumore,
ma al contrario si ha una maggior capacit\a di localizzazione dei bordi \e la possibilit\a
di ottenere rappresentazioni compatte del contorno dei segmenti, in particolare con l'utilizzo
di opportuni modelli di {\it contorni attivi} ({\it active contours} o {\it snakes}) che sono
trattati in questa tesi.

Recentemente sono stati proposti inoltre degli algoritmi con combinano i due
approcci (\cite{Duncan}, \cite{Yuille} e \cite{Yezzi}) in modo da compensarne i difetti e
sommarne i vantaggi.
La prima soluzione \cite{Duncan} prevede l'interazione dei due blocchi distinti attraverso
una serie di regole che stabiliscono come i due interagiscono fino a raggiungere uno stato di
equilibrio.

La seconda soluzione, che \e qui considerata, realizza l'obiettivo a partire dalla
definizione di un nuovo modello di snakes in cui i contorni non sono direttamente
caratterizzati come le zone ad alto gradiente, ma indirettamente attraverso
una nuova funzione energia che tiene conto della statistica delle regioni separate dallo
snake e che \e minima quando questi si dispone lungo il contorno dell'oggetto considerato.
In particolare l'algoritmo proposto da Yezzi et al. in \cite{Yezzi} presenta delle specifiche
che lo rendono interessante per la semplicit\a della definizione della funzione energia e 
per ci\o che ne consegue in termini di implementazione.

L'algoritmo permette infatti di segmentare l'immagine nei suoi diversi elementi non appena
sia fissata la primitiva, {\it feature}, rispetto alla quale risultano distinti; \e da
notare che non \e necessario conoscere il valore della primitiva caratteristico di ciascun
oggetto ma \e sufficiente sapere che per ciascuno \e differente.
Un'altro vantaggio rilevante \e dato dal fatto che la definizione data permette di considerare
non una sola feature alla volta, ma contemporaneamente un vettore di features che possono
essere di natura diversa tra loro.
Nel caso specifico permette di trattare le immagini a colori che sono matrici $R \times C$ di 
pixels a ciascuno dei quali \e associato un vettore di tre elementi, nel caso pi\u semplice
le tre componenti $RGB$.

Accanto alla definizione della funzione energia occorre definire il modello che 
rappresenta fisicamente la curva chiusa e quindi lo snake: si \e scelta
una rappresentazione a B-splines cubiche uniformi.
La curva \e rappresentata perci\o come l'unione di tratti di cubica ({\it spans}) che sono
combinati secondo dei parametri, i punti di controllo o {\it control points}; l'ordine del
polinomio definisce il grado di regolarit\a della curva.

In tal modo si \e ottenuta una rappresentazione compatta, come si voleva, 
interamente rappresentata dai punti di controllo, che sono in numero limitato, e rispetto
ai quali \e possibile definire le misure delle caratteristiche geometriche del contorno.

Un problema associato alla scelta del metodo di segmentazione riguarda l'inizializzazione
del contorno attivo: lo snake definito in \cite{Yezzi} concettualmente \e 
abbastanza insensibile all'inizializzazione (in pratica vi possono essere dei limiti se
l'immagine non \e considerabile bimodale) a differenza dei modelli di snake precedenti che
si basano sulla ricerca degli edge nell'intorno della curva e per i quali \e opportuna una
buona inizializzazione.
In ogni caso \e opportuna per ridurre i tempi di esecuzione dell'algoritmo.

L'inizializzazione si pu\o intendere come una fase di presegmentazione ed \e data dal
contorno della regione ricavata binarizzando la componenete principale della
{\it trasformata di Karhunen-Lo\`eve} (o di {\it Hotelling}) dell'immagine originale, seguita
da alcune elaborazioni con operatori morfologici per ottenere una regione compatta.
Inizializzando la curva in prossimit\a del bordo della macchia \e possibile realizzare
un'implementazione locale dell'algoritmo di Yezzi che permette di ridurre le due regioni
interna ed esterna a due fascie definite nell'intorno dello snake.
In tal modo si ha che le due aree possono essere approssimativamente uguali, indipendentemente
dalle dimensioni della macchia rispetto al piano immagine, ed inoltre \e
possibile escludere le zone prossime ai bordi del sensore, maggiormente sensibili ai disturbi
dovuti alla illuminazione non uniforme e agli stessi effetti di bordo dell'ottica.

In questa fase \e possibile inoltre ridurre gli elementi di disturbo come i peli: 
sfruttando il fatto che questi hanno una elevata correlazione lungo una direzione, che
\e comunque incognita, \e possibile realizzare un operatore locale che
elabora l'immagine in funzione della misura di anisotropia rilevata nell'intorno
del punto considerato. 
Esistono diverse definizioni per tale misura, nel nostro caso si \e considerata una funzione
della media dei valori dell'intensit\a della stessa componente principale lungo
direzioni\footnotemark\, prefissate; quindi in corrispondenza dei punti che
si considerano appartenenti al pelo, per cui la misura \e superiore ad un valore di soglia,
si sostituisce il valore dell'intensit\a con un nuovo valore che dipende dai pixels
circostanti che molto probabilmente appartengono alla cute o al neo.

\footnotetext{Considerando che tali rette-direzioni sono definite in uno spazio non
 continuo ma discreto (segmenti di rette digitali).}

A partire dalla curva iniziale, definita da un certo numero di control points distribuiti
nel piano, \e necessario permettere che questa possa adattarsi alla complessit\a 
dell'oggetto che rileva durante la sua evoluzione.
Si \e pensato quindi di realizzare una riparametrizzazione della curva, basata su un
metodo di {\it triangolazione} (o {\it scomposizione simpliciale}), e la definizione
di un criterio di complessit\a per determinare un nuovo numero di punti di controllo.
 

%--------------------------------------------------------------------------------------------
% Contenuto dei capitoli
\contcap{Contenuto dei capitoli}

\begin{description}
\im[\sl Capitolo 2.] Sono descritti alcuni operatori per l'elaborazione digitale delle
   immagini:
   \bi
   	\im {\it trasformata di Karhunen-Lo\`eve}
   	\im {\it operatori morfologici}
   	\im {\it rank-order filters}
   \ei	
\im[\sl Capitolo 3.] Introduzione al problema della segmentazione e descrizione di alcuni
   algoritmi di binarizzazione.
\im[\sl Capitolo 4.] Definizione e propriet\a delle B-splines.
\im[\sl Capitolo 5.] I modelli dei contorni attivi: l'approccio statistico proposto da
   Yezzi et al. \cite{Yezzi}.
\im[\sl Capitolo 6.] Descrizione dell'architettura dell'algoritmo proposto con alcuni
   risultati.
\end{description}

\finepar